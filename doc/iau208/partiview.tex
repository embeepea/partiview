% $Log: partiview.tex,v $
% Revision 1.11  2002/07/15 21:42:58  slevy
% Add two short references.
%
% Revision 1.10  2002/07/10 23:09:32  slevy
% Smaller images; minor changes, including dealing with Piet and Peter's
% comments.  Give a www reference for Virdir.
%
% Revision 1.9  2002/07/03 17:37:09  slevy
% Change right-hand page heading to "Partiview".
%
% Revision 1.8  2002/07/03 13:47:05  slevy
% Make figures fit better, at least on 8.5x11" page.
%
% Revision 1.7  2002/07/03 13:32:26  slevy
% Merge figure-size changes.
%
% Revision 1.6  2002/07/03 13:27:17  slevy
% Draft of complete paper.
%
% Revision 1.5  2002/06/29 07:03:07  slevy
% ...
%
% Revision 1.4  2002/06/29 06:50:29  slevy
% Aha, so that's what \markboth means.
% Toss duplicate figures.
%
% Revision 1.3  2002/06/29 06:18:03  slevy
% Don't try to put a pair of figures together in a row;
% centerline{} isn't enough to allow captions to wrap.
% Add Offerings and Thanks.
% More about starlab.  Need more yet.  Maybe one more figure.
%
% Revision 1.2  2002/06/28 01:41:59  slevy
% More written.
%
% Revision 1.1  2002/06/27 01:49:30  slevy
% A bit more, including png images for two of four figures.
%
% Revision 1.1  2002/06/26 19:31:53  slevy
% Maybe this will turn into an IAU208 paper about partiview.
% Figures to appear in same partiview/doc/iau208 directory.
%
\documentstyle[11pt,newpasp,twoside,psfig]{article}
\markboth{Stuart Levy}{Partiview}
\pagestyle{myheadings}
\nofiles

% Some definitions I use in these instructions.

\def\emphasize#1{{\sl#1\/}}
\def\arg#1{{\it#1\/}}
\let\prog=\arg

\def\edcomment#1{\iffalse\marginpar{\raggedright\sl#1\/}\else\relax\fi}
\marginparwidth 1.25in
\marginparsep .125in
\marginparpush .25in
\reversemarginpar

\begin{document}
\title{Interactive 3-D visualization of particle systems with Partiview}
 \author{Stuart Levy}
\affil{National Center for Supercomputing Applications, 
        University of Illinois Urbana-Champaign,
	605 E. Springfield,
	Champaign, IL 61820}

\begin{abstract}
This paper describes partiview, a free software tool for 
interactive graphical display of collections of particles
in 3-space, and its application in studying the results
of N-body collisional stellar dynamics calculations from Starlab.
\end{abstract}

% ---

\section{Partiview}
Partiview is an interactive graphical software tool, focused on 
flexible display of particles in 3-space.

Input data to partiview is provided as a (possibly time-varying)
collection of particles, each with a 3-D position and an arbitrary number
of other floating-point attributes ({\it fields}), and a configuration script
specifying which fields to map into visible properties,
including color and luminosity.  Text-based commands can change
these selections interactively, switching between coloring stars by
e.g. temperature, mass, or space velocity on the fly.
For example, if a field were named Tlog,
{\verb"  color Tlog 3.2 4.5
"}
assigns colors by using the Tlog field as an index into a
user-supplied color table via a linear mapping
that associates 3.2 and 4.5 with the colormap's endpoints.

From each particle's luminosity and distance from the current
viewpoint, partiview draws a dot whose screen brightness and size
suggest its computed apparent brightness.  With dots up to a few
pixels across, apparent brightness may usefully range by several
hundredfold, and larger ranges can be suggested by adding textured
polygons -- ``haloes'' -- whose size varies similarly.  The result
is good enough to yield plausible naked-eye starfields given a table
of stellar luminosities, colors and 3-D positions as in Figure~1,
drawn using Hipparcos data.

This kind of viewpoint-dependent apparent brightness
is a feature that few other scientific visualization
packages seem to offer, even though it's
inexpensive to compute and can be useful.  
(Where not useful, as when making orthographic
plots of 3-D scenes, it can be switched off in partiview.)

Some database-like operations are provided.
For example, one can display only the subset of particles where
some (single) field has values in a given range or set,
or look only at particles lying within a given rectangular subvolume,
Also one can print a histogram of values of a field,
over all particles or the selected subset.

\begin{figure}
  \psfig{figure=starfield.ps,height=3.2in}
  \caption{Desktop version of partiview showing star field
    from Hipparcos data with Sun at upper left, marked by 0.1 pc crosshair.
    Normally displayed in light colors on a dark background, this and all
    images in this article are rendered as black-on-white for publication.}
\end{figure}

\begin{figure}[htb]
  \psfig{figure=ncsa-haydenCollab2.ps,height=3.2in}
  \caption{Virtual Director version: collaborating between
    Hayden Planetarium (dome avatar at lower left) and NCSA while
    designing an animation path in a simplified Milky Way model.}
\end{figure}

\section{Scripting}

Partiview can be controlled by external scripts in a limited way,
by invoking a subprocess which emits a series of partiview commands.
This allows writing, for example, animation-recording programs which
drive partiview to display some computed sequence of views and record
each image to a file.  It may be adequate for creating simple
graphical controls for features of partiview that aren't
provided on the existing graphical panel.  The major limitation
is the lack of feedback: there's no way for external driver programs
to read out the viewer's state.

A future version of the software may have an embedded scripting language,
perhaps Python or Ruby.  This should make it easier to offer a variety
of graphical interfaces (or to change the graphical interface dynamically),
to couple the viewer more easily with external data-mining tools, and so on.

\section{Desktop and dome}

The same graphical and data-handling code is embedded
in multiple guises for different computing environments.
Both accept the same data and configuration files, and most of the same
text-based interactive commands.  Figure~1 illustrates the
desk- (or lap-)top version, mouse and keyboard driven
with conventional buttons and sliders for common controls,
available for Unix-like systems and for Windows.
Figure~2 shows the virtual-reality version,
built using the Virtual Director virtual-choreography framework
$<$http://virdir.ncsa.uiuc.edu/virdir/$>$ 
and the CAVE library $<$http://www.evl.uic.edu/pape/CAVE/$>$;
it is currently restricted to Silicon Graphics computers
but, unlike the desktop implementation of partiview,
it can run on systems with multiple graphics pipes.

Though the latter was originally written for the
CAVE virtual reality room at NCSA, it is used elsewhere as well.
The Hayden Planetarium at the American Museum of Natural History
in New York built a Silicon Graphics-driven display for their
planetarium dome; this turned out sufficiently
CAVE-like that the same software runs in the Hayden dome and is
regularly used there.

The Virtual Director framework supports networked collaboration too,
allowing widely-separated participants to exchange viewpoints
(represented by ``avatars''), animation paths, display settings
and so on.  This has proven helpful for bringing together distributed
expertise, and we've used it between Illinois, Hawaii, New York
and elsewhere on several occasions, most recently when designing
some animations for Hayden's 2002 space show, as seen in Figure~2.


\section{N-body dynamics: examining Starlab traces}

Stellar dynamics simulations done in Starlab $<$http://www.manybody.org/$>$
produce {\it traces} as output.  Trace files record various information
about each star as a function of time: physical properties such as mass,
luminosity and temperature; position and three time derivatives;
and hierarchical descriptions (binary trees) of interacting groups.

Partiview, coupled with the Starlab libraries to read and interpolate
traces, is adapted to display these properties as the cluster evolves.

Figure~3 shows a cluster evolving in a tidal field.
``Trails'' show the recent motion
history of each star, so long trails show high speed, and curved ones
high acceleration.  Note the stars escaping through the tidal tails
at left and right.

\begin{figure}
  \psfig{figure=0607.ps,height=3.0in}
  \caption{Star cluster dispersing in a tidal field,
    with recent motion shown by trails.  Small circles denote binary or
    multiple systems.}
\end{figure}

\begin{figure}
  \psfig{figure=primbin16.ps,height=2.5in}
  \caption{Interacting groups of stars from Starlab.}
\end{figure}

\section{Microscopy}
A sort of interaction microscope is shown in figure~4 on a tiny test case.
Bound or strongly-interacting systems are circled, and colored according to the
number of members of the group.  Circle sizes are related
to separation or semimajor axis, but can be constrained to a range
of screen sizes to ensure that even tightly-bound systems are always
visibly marked.   The binary trees of interacting groups,
with stars as leaves and center-of-mass points as nonleaf nodes,
are shown as stick figures.
Tick marks perpendicular to the tree branches, drawn in the screen plane,
serve a dual purpose: their positions show the center-of-mass location,
and their lengths are proportional to the instantaneous true separation
of the nodes on that branch.

Note the triple system in figure~4.  Although in this view its
lower-right pair of stars seem very close together, this must be a 
projection effect; as shown by the tick marks, their true separation
is nearly as large as that from their center-of-mass to the upper-left star.

% Each star's dynamical state is sufficiently finely sampled in time
% to allow accurate interpolation, generally at some fixed multiple of the
% internal simulation timestep.   Thus stars in dense regions
% may have far more frequent trace entries than isolated stars.
% The Starlab libraries offer functions to interpolate the state of the
% simulation at any time.

Stellar cluster simulations, with time scales spanning many orders
of magnitude, need some sort of temporal microscope too.  Partiview has
limited facilities for this, with a logarithmic speed scale
(simulation time per unit displayed time), keystroke commands to
home in on interesting events, and a trip meter to mark a (single)
time reference point.  


\section{Sifting needles from a haystack}
In all but the smallest systems, direct visual inspection quickly becomes
an impractical way to study interaction patterns -- too many stars
clutter the view.  But visualization might still be helpful, if there is a way
to focus on a subset of the system that sheds light on the question at hand.

For example, to study some sequence of events -- the ejection of a star
from the cluster, or the formation of a hard binary -- one might
imagine focusing by looking at some limited neighborhood,
e.g. all stars passing through some volume during some time interval.
But, since starlab simulation traces explicitly mark
closely-interacting stars, we can do better than that; we can 
use {\it interaction tracking} to pick out just the set of stars that
pass close enough to be dynamically important.

To do this, the user constructs an initial small
{\it selection-set} $S$ of interesting stars at a given time -- perhaps by
a threshold of some field (``all stars with mass exceeding 30 $M_{sun}$'', or
``all triple systems'', or ``star number 1792''), or by clicking on some star,
e.g. an escapee.  This set is given to a {\verb"kira intsel"}~$T$~=~$S$
command, and the user plays the trace forward or backward in time
for a while.  When {\verb"intsel"} is in effect, whenever any member of
$S$ interacts with another star, the software adds the latter star to the set $T$.
If $S$ and $T$ refer to the same set, then that set accumulates the transitive
closure of interactions with $S$.

Other commands can then use membership in $T$ to
affect the display: to show only stars which have
or haven't interacted during the given time span; to show all stars,
but brightening those in $T$; to draw trails behind those selected; etc.

\section{Other views}
Of course, plots of 3-D positions aren't the only
useful way to display the state of a cluster.  For Starlab output traces which
record stellar temperature and luminosity, a Hertzprung-Russell
diagram is also available, as shown in Figure~5.  One can click on a star
in either view and locate it in the other.

\begin{figure}
\psfig{figure=hyades-hrdiag.ps,height=2.1in}
\caption{H-R diagram}
\end{figure}

Other types of views are contemplated for future work;
for example, worldline plots of two spatial dimensions plus time,
or a velocity-space plot.

\section{Offerings}
The desktop version of partiview is available as free software from
the Starlab CVS archive; see Peter Teuben's web pages at:\hfill\break
$<$http://www.astro.umd.edu/nemo/amnh/$>$.

As described in Chandler (2002),
Brian Abbott and Carter Emmart of the American Museum of Natural
History have assembled a 3-D Milky Way model (nearby stars,
open and globular clusters, H~II regions, pulsars, etc.) from various
sources and are offering it in partiview form, along with partiview itself
for Linux and Windows, on the Hayden Planetarium web site:\hfill\break
$<$http://www.haydenplanetarium.org/hp/vo/partiview/index.html$>$.

\section{Acknowledgements}
Thanks are due in many directions: to my colleagues Donna Cox and
Robert Patterson at NCSA who've encouraged me to develop this;
to them and other patient users, Emmart and Abbott of Hayden/AMNH; Brent Tully
of U. of Hawaii; Piet Hut, Peter Teuben, Steve McMillan, Simon
Portegies-Zwart, and Jun Makino of the Starlab group.  All have offered data
to visualize and plenty of good ideas, a few of which are implemented here.
Special thanks are due to Teuben, who wrote much of the extant
documentation, and to McMillan for the Starlab interface libraries.
Thanks finally to the NCSA, which is supported in turn
by the National Science Foundation and by the state of Illinois.

\begin{thebibliography}
\bibitem[2002]{Hut2002}
{Hut}, P. 2002, these proceedings.

\bibitem[2002]{Chandler2002}
{Chandler}, David L. 2002, \skytel August 2002, 61
\end{thebibliography}

\end{document}
